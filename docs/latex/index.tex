\href{https://travis-ci.org/zwimer/DrShadowStack}{\tt !\mbox{[}Build Status\mbox{]}(https\-://travis-\/ci.\-org/zwimer/\-Dr\-Shadow\-Stack.\-svg?branch=master)}

Dr\-Shadow\-Stack is a software defined dynamic shadow stack implemented via Dynamo\-R\-I\-O. Dr\-Shadow\-Stack implements a shadow stack any binary given to it, provided the file has an E\-L\-F header. If the program attempts to return to a corrupted return address, Dr\-Shadow\-Stack will terminate the entire process group (which it sets up). Dr\-Shadow\-Stack can handle multi-\/threaded processes, processes that fork, processes that call any variation of exec. All of these processes will also be protected by Dr\-Shadow\-Stack.

\section*{Table of Contents}


\begin{DoxyEnumerate}
\item \href{#docker}{\tt Docker}
\end{DoxyEnumerate}
\begin{DoxyEnumerate}
\item \href{#requirements}{\tt Requirements}
\end{DoxyEnumerate}
\begin{DoxyEnumerate}
\item \href{#installation-instructions}{\tt Installation Instructions}
\end{DoxyEnumerate}
\begin{DoxyEnumerate}
\item \href{#usage}{\tt Usage}
\end{DoxyEnumerate}
\begin{DoxyEnumerate}
\item \href{#example}{\tt Example}
\end{DoxyEnumerate}
\begin{DoxyEnumerate}
\item \href{#documentation}{\tt Documentation}
\end{DoxyEnumerate}
\begin{DoxyEnumerate}
\item \href{#developers}{\tt Developers}
\end{DoxyEnumerate}

\subsection*{Docker}

A {\ttfamily Dockerfile} is provided with a pre-\/installed {\ttfamily Dr\-Shadow\-Stack} binary. A docker image is also provided. It is hosted \href{https://cloud.docker.com/u/zwimer/repository/docker/zwimer/drshadowstack}{\tt here} on \href{dockerhub.com}{\tt dockerhub.\-com}. To pull the docker image simply execute\-: ```bash docker pull zwimer/drshadowstack ``` To run the container simply execute\-: ```bash docker run --rm -\/it zwimer/drshadowstack ``` If you would like to build the container yourself execute\-: ```bash git clone \href{https://github.com/zwimer/DrShadowStack}{\tt https\-://github.\-com/zwimer/\-Dr\-Shadow\-Stack} \&\& \textbackslash{} cd Dr\-Shadow\-Stack \&\& \textbackslash{} docker build -\/t zwimer/\-Dr\-Shadow\-Stack . ```

\subsection*{Requirements}


\begin{DoxyEnumerate}
\item This project utilizes \href{https://github.com/DynamoRIO/dynamorio}{\tt Dynamo\-R\-I\-O} version {\ttfamily 7.\-0.\-17636}. This release can be found \href{https://github.com/DynamoRIO/dynamorio/releases/download/cronbuild-7.0.17636/DynamoRIO-x86_64-Linux-7.0.17636-0.tar.gz}{\tt here}.
\item This project utilizes the C++ library \href{https://boost.org}{\tt Boost}. This library can be built from source, as explained \href{https://www.boost.org/doc/libs/1_66_0/more/getting_started/unix-variants.html}{\tt here}. On \href{http://releases.ubuntu.com/16.04.4/}{\tt Ubuntu 16.\-04}, this library can be installed as follows\-: ```bash sudo apt-\/get update \&\& sudo apt-\/get install libboost-\/all-\/dev ``{\ttfamily }
\item {\ttfamily The project is built on \mbox{[}Ubuntu 16.\-04 L\-T\-S\mbox{]}(\href{http://releases.ubuntu.com/16.04.4/}{\tt http\-://releases.\-ubuntu.\-com/16.\-04.\-4/}) via \mbox{[}C\-Make\mbox{]}(\href{https://cmake.org/}{\tt https\-://cmake.\-org/}). It requires compiler that supports}C++11`.
\item The system architecture is in the x86 or x86\-\_\-64 families.
\end{DoxyEnumerate}

For more specific information about requirements, visit the requirements wiki page \href{https://github.com/zwimer/DrShadowStack/wiki/Requirements}{\tt here}.

\subsection*{Installation Instructions}


\begin{DoxyEnumerate}
\item Install dependencies
\item Clone the repository ```bash git clone \href{https://github.com/zwimer/DrShadowStack}{\tt https\-://github.\-com/zwimer/\-Dr\-Shadow\-Stack} ``{\ttfamily }
\item {\ttfamily Configure the}Dr\-Shadow\-Stack/src/\-C\-Make\-Lists.\-txt` file. Instructions in the file itself.
\item Create a build directory ```bash cd Dr\-Shadow\-Stack/src mkdir build \&\& cd build ```
\item Build with C\-Make and make ```bash cmake .. \&\& make -\/j 4 ```
\end{DoxyEnumerate}

\subsection*{Usage}

The full usage of this program can be found via\-: {\ttfamily ./\-Dr\-Shadow\-Stack -\/-\/help}

In general, the usage is of this format\-: ```bash ./\-Dr\-Shadow\-Stack \mbox{[}--ss\-\_\-mode $<$\-Mode$>$\mbox{]} $<$executable target$>$=\char`\"{}\char`\"{}$>$ $<$target arguments$>$=\char`\"{}\char`\"{}$>$ ```

There are two different modes, {\ttfamily int} (internal) and {\ttfamily ext} (external). The internal mode keeps the shadow stack internally in the Dynamo\-R\-I\-O client. The external mode stores the stack in a separate process.

\subsection*{Example}

From the build directory of a previous version, an example could be\-: ```bash vagrant-\/xenial $\sim$/\-S/s/build$>$ ./\-Dr\-Shadow\-Stack ls -\/la ./ total 612 drwxrwxr-\/x 3 vagrant vagrant 4096 Apr 3 20\-:01 . drwxrwxr-\/x 4 vagrant vagrant 4096 Apr 3 19\-:59 .. -\/rw-\/rw-\/r-- 1 vagrant vagrant 14536 Apr 3 19\-:59 C\-Make\-Cache.\-txt drwxrwxr-\/x 7 vagrant vagrant 4096 Apr 3 20\-:01 C\-Make\-Files -\/rw-\/rw-\/r-- 1 vagrant vagrant 1381 Apr 3 19\-:59 cmake\-\_\-install.\-cmake -\/rw-\/rw-\/r-- 1 vagrant vagrant 8045 Apr 3 20\-:01 compile\-\_\-commands.\-json -\/rwxrwxr-\/x 1 vagrant vagrant 402168 Apr 3 20\-:01 Dr\-Shadow\-Stack -\/rwxrwxr-\/x 1 vagrant vagrant 53264 Apr 3 20\-:00 libss\-\_\-dr\-\_\-client.\-so -\/rwxrwxr-\/x 1 vagrant vagrant 96304 Apr 3 19\-:59 libss\-\_\-support.\-so -\/rw-\/rw-\/r-- 1 vagrant vagrant 15265 Apr 3 20\-:01 Makefile -\/rw-\/rw-\/r-- 1 vagrant vagrant 8157 Apr 3 20\-:01 ss\-\_\-dr\-\_\-client.\-ldscript ```

\subsection*{Documentation}

Additional documentation of Dr\-Shadow\-Stack can be found in the \href{https://github.com/zwimer/DrShadowStack/wiki}{\tt wiki}.

\subsection*{Developers}

Before pushing any code, please run the {\ttfamily run-\/before-\/push.\-sh} script. This will automatically update the changelog and format all {\ttfamily C++} code.

Additional documentation to each component of Dr\-Shadow\-Stack is built automatically via \href{https://travis-ci.org/}{\tt Travis C\-I} utilizing \href{http://www.stack.nl/~dimitri/doxygen/}{\tt Doxygen}, and hosted \href{https://zwimer.com/DrShadowStack}{\tt here}. 